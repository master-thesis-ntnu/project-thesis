\chapter{Introduction}
\section{Motivation}
[Maby mention something about todays search engines?]

Today an increasing number of software companies try to personalize their services.
A few examples of personalized services are Facebook's news feed, Netfix's movie suggestion and Spotifi's Discover Weekly.

In research done by Google \cite{google-latency} they found that a latency of 100ms or less is required for interactive tasks.
As a result search engines most deliver results back to the user within 100ms, or else the user is most likely to abondon the website.

This project thesis is based on the work done in the master thesis \textit{Instant, Personalized Search Recommandation} by Juul Arthur R Rudihagen \cite{master-thesis}.
In the master thesis Rudihagen researched how an instant personalized search could be achieved.
The research achieved a personalized search.
However, several of the search implemantations exceeded the limit of 100ms, as stated in the master thesis by Rudihagen \cite[sec 6.6]{master-thesis}.

This project thesis examines how a fast and scaleable search may be achieved, with the research done by Rudihagen as a baseline.
His search implementations had limits towards latency and scalability.
A major part of the latency came from multiple round trips from the webserver to the database and the search engine.

\section{Problem specification}
This project thesis tries to answer the research questions below, with the main focus being research question \ref{rq:scaling} and \ref{rq:latency}.

\begin{enumerate}
  \item How to provide instant personalized recommendations on a cold start based on the query typed?
  \item Does the personalized recommendations give better results for the user?
  \item\label{rq:scaling} How to make the search recommandation scale with an increasing amount of data?
  \item\label{rq:latency} How to provide search recommendations within 100ms?
\end{enumerate}

\section{Structure}
Chapter \ref{ch:background} \\
Chapter \ref{ch:related-work} \\
Chapter \ref{ch:approach} \\
Chapter \ref{ch:evaluation} \\
Chapter \ref{ch:conclusion}
