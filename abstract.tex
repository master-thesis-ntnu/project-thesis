\chapter*{Abstract}
Today's digital services generate an increasing amounts of data.
With the large amounts of data, arrives an issue on which results should be regarded as relevant for the user.
The results should not only be relevant, but also arrive fast.
Results have to arrive within 100 ms for interactive tasks like search.

Query expansion may be used to aid the user when searching,
but extracting relevant information to use with query expansion may sometimes be a challenge.
A technique called pseudo-relevance feedback have shown to provide good results in terms of relevance.
There are a considerable amount of research on pseudo-relevance feedback and query expansion.
However, the focus is rarely on speed and scaleability.

With the increasing amounts of data, scalable implementations are important to handle the all the data.
Speed has proven to be an important factor for web services.
The company Amazon even states that 100 ms in latency leads to 1\% in earnings.

This project report utilize the search engine Elasticsearch,
comined with a techique called pseudo-relevance feeback to implement query expansion.
We will see that the latency results show great potential, and may be used with interactive tasks,
even when the amount of data increases.
