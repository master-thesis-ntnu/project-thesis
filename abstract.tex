\chapter*{Abstract}
Today's digital services generate increasing amounts of data.
With the large amounts of data, arrives an issue on which results should be regarded as relevant for the user.
The results should not only be relevant, but also arrive fast.
Results have to arrive within 100 ms for interactive tasks.
Speed has been proven to be an important factor for digital services.
The company Amazon even states that 100 ms in latency on their webpage leads to 1\% loss in earnings.

To aid the user when search, a technique called query expansion may be used.
Extracting relevant information to use with query expansion may sometimes be a challenge.
To overcome this challenge a method called pseudo-relevance feedback have shown to provide good results in terms of relevance.
There are a considerable amount of research on pseudo-relevance feedback and query expansion.
However, the focus is rarely on speed and scalability.

Most of the common available technologies today have limited or no support for query expansion.
This project report will explore how the search engine Elasticsearch can be used to implement query expansion.
During this report we will also measure how much impact query expansion have on latency for a web service,
compared to a baseline query.
