\chapter{Conclusion \& Futher Work}
\label{ch:conclusion}
This chapter discuss the query expansion implementation and compares it to Rudihagen's implementation.
Lastly, possible improvements are suggested to futher increase performance in terms of latency and relevance.

\section{Conclusion}
This report explored the possibilities to achieve instant search with query expansion.
The implementation involves the search engine Elasticsearch, as it has proven to scale with large amounts of data.

From the results we see that the implementation


\section{Futher Work}
After implementing a query expansion which requires two round trips there is still room for improvement.
To achieve only one round trip, the two following solutions should be explored:
use Elasticsearch's scripting functionality to evaluate custom expressions\footnote{\url{https://www.elastic.co/guide/en/elasticsearch/reference/current/modules-scripting.html}},
or implement query expansion in Lucene and expose the feature through Elasticsearch's API.

There are photos from the dataset which doesn't contain any tags,
and the current implementation only searches the tag field.
This constraint means that photos without any will have the lowest document score, and thus never be visible in the results.
To handle the problem an improved implementation should use more text features as title and description.

Evaluating the
[real world setup]

As mentioned earlier the document relevance was not measured.
In further work, the result's relevance should be measured, too assure that the query expansion indeed increases result relevance.


[Using more features e.g. location and time]
Implement other query expansion techniques rochio
Use query expansion from implicit data.
