\chapter{Background}
\section{Query Expansion}

\section{Technology}
During testing this projec thesis utilized a webserver and a search engine.
NodeJS \footnote{\url{https://nodejs.org}} v7 were chosen as the webserver.
NodeJS were primarly chosen because the author of this project thesis had used it in multiple projects before.
Inside NodeJS lies the V8 \footnote{\url{https://developers.google.com/v8/}} JavaScript engine.
With NodeJS comes a rich package system delivered through NPM.

Elasticsearch \footnote{\url{https://www.elastic.co/products/elasticsearch}} v5 were utilized as the search engine.
At the time of writing Elasticsearch is one of the most popular search engine.
Elasticsearch is open source and built on top of Lucene \footnote{\url{https://lucene.apache.org/}}.
Lucene is the search engine itself,
and Elasticsearch provides functionality for distribution and a REST API interface.
