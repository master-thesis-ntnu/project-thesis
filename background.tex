\chapter{Background}
\label{ch:background}

\section{Technology}
During testing this projec thesis utilized a webserver and a search engine.
NodeJS \footnote{\url{https://nodejs.org}} v7 were chosen as the webserver.
NodeJS were primarly chosen because the author of this project thesis had used it in multiple projects before.
Inside NodeJS lies the V8 \footnote{\url{https://developers.google.com/v8/}} JavaScript engine.
With NodeJS comes a rich package system delivered through NPM.

Elasticsearch \footnote{\url{https://www.elastic.co/products/elasticsearch}} v5 were utilized as the search engine.
At the time of writing Elasticsearch is one of the most popular search engine.
Elasticsearch is open source and built on top of Lucene \footnote{\url{https://lucene.apache.org/}}.
Lucene is the search engine itself,
and Elasticsearch provides functionality for distribution and a REST API interface.

\section{Search Engine (Elasticsearch)}
This section describes how Elasticsearch scores its documents and is based on the documentation found on the website \cite{elasticsearch-scoring}.
The following subsections are needed to understand how the end result is determined, when using a term search in Elasticsearch.

\subsection{Term Frequency}

\subsection{Inverse Document Frequency}

\subsection{Field Length Norm}

\subsection{Lucene Scoring Function}
\url{http://lucene.apache.org/core/3_0_3/api/core/org/apache/lucene/search/Similarity.html}
\begin{cequation}[H]
	\begin{equation}
		\mathbf{score(q,d)} = coord(q,d) * queryNorm(q) * sum [tf(t in d) * idf(t)^2 * t.getBoost() * norm(t,d)]
	\end{equation}
	\caption{Lucene Scoring Function}
  \label{eq:scoring-function}
\end{cequation}

\section{Query Expansion}
[Write about Query expansion]
Often the user's search is imprecise.
To aid the user when search

Kullback-Leibler divergence measures how well distribution P represents the distrubution Q.


% Kullback-Leibler Distance Equation
\begin{cequation}[H]
	\begin{equation}
	    \label{equ:line}
		\mathbf{KL}_D[P(t), Q(t)] = P(t)*\log{[\frac{P(t)}{Q(t)}]}
	\end{equation}
	\caption{Kullback-Leibler Distance}
  \label{kl-distance}
\end{cequation}
