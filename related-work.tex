\chapter{Related Work}
\label{ch:related-work}

\section{Twitter Query Suggestion Engine}

Twitter’s query suggestion engine \cite{twitter-suggestion} returns the most trending topics and merge this result with more long term trending topics.
Twitter have low latency requirements, however the query suggestion is not personalized in the same way as this project thesis tries to achieve.

\section{Fuzzy Search}
A studdy found that 50\% of users reformulate their queries, and close to a third of these users reformulated their query three times or more \cite{query-reformulate}[This should be moved to a different section].
With a standard term search the user has to spell the term correct, a misspelled term will most likely yield no or few results.
E.g the term "blu" should also retrieve results on the term "blue."
To handle this problem a technique called fuzzy search may be used.

A commonly used method with fuzzy search is Levenshtein distance.
Levenshtein distance calculates the number of characters edits which are required to transform one string into another.
Editting a string includes substitution, insertion and deletion.
Levenshtein distance can be expanded with the method Damerau to allow character transposition.

Elasticsearch uses Damerau-Levenshtein distance to calculate edit distance \footnote{\url{https://www.elastic.co/guide/en/elasticsearch/guide/current/fuzziness.html}}.
The Damerau-Levenshtein distance corresponds to the number of character edits.
A distance of 1, means one character has to be changed.

\section{Instant, Personalized Search Recommandation}
An earlier master thesis on this topic implemented a recommendation engine which was able to give better search recommendations.
However the implemented solution experienced a latency of up to 600 ms.
In this thesis we will explore the possibility to reduce the search latency to beneath the 100 ms mark. [Instant, personalized search recommendations]

\section{Pseudo-Relevance Feedback}
Section \ref{sec:query-expansion} how query expansion work and that data are required to measure relevance.
Often the top-k documents are often used to find pseudo-relevance for query expansion.
However, the top-k documents are in many case not relevant to be used for query expansion \textit{Selecting good expansion terms for pseudo-relevance} \cite{pseudo-relevance-invalid}. 

\section{Relevance Feedback}

\subsection{Explicit Feedback}

\subsection{Implicit Feedback}
